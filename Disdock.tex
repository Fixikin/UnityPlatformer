\documentclass[a4paper,12pt]{article}
\usepackage[utf8]{inputenc}
\usepackage[russian]{babel}
\usepackage{geometry}
\geometry{a4paper,left=20mm,right=20mm,top=20mm,bottom=20mm}
\usepackage{enumitem}
\usepackage{array}
\usepackage{graphicx}
\usepackage{float}

\title{Дизайн-документ}
\author{Скачков Юрий}
\date{Январь 2025}

\begin{document}

\maketitle

\tableofcontents

\newpage

\section{Введение}

Данный документ представляет собой концепцию игры \textbf{``Run Boy Run''}, описывающую ее основную идею, механику, жанр, целевую аудиторию и ключевые особенности. Документ структурирован по разделам, позволяющим получить полное представление о проекте.

\begin{itemize}
    \item \textbf{Ссылки на используемые материалы}: [Ссылки на материалы, если есть]
    \item \textbf{История изменений документа}: [История изменений, если есть]
    \item \textbf{Список авторов}: Скачков Юрий
    \item \textbf{Условные обозначения и сокращения}: В документе используются стандартные термины и сокращения, принятые в игровой индустрии. Названия игры, персонажей, мест выделены кавычками.
\end{itemize}

\newpage

\section{Концепция}

\subsection{Введение}

\textbf{``Run Boy Run''} — это динамичный платформер с элементами стелса и экшена, где игрок управляет подростком, обладающим важной информацией против преступного синдиката. Цель игры — выбраться из города живым, избегая преследователей и преодолевая различные препятствия. Игра предлагает напряженный геймплей, где каждый шаг может стать последним, а каждый уровень — новым вызовом.

\subsection{Жанр и аудитория}

\begin{itemize}
    \item \textbf{Жанр}: 2D-платформер с элементами стелса и экшена.
    \item \textbf{Целевая аудитория}: Игроки от 12 лет и старше, которые любят динамичные игры с напряженной атмосферой и интересным сюжетом.
    \item \textbf{Позиционирование}: Игра подходит как для любителей платформеров, так и для тех, кто ищет захватывающий сюжет и вызов.
\end{itemize}

\subsection{Основные особенности игры}

\begin{itemize}
    \item Динамичный геймплей с элементами стелса и погонями.
    \item Разнообразные уровни с уникальными препятствиями и врагами.
    \item Возможность использовать окружение для побега (укрытия, ловушки).
    \item Эмоциональный сюжет, который держит игрока в напряжении.
    \item Примерный объем игры: 5-10 минут прохождения.
\end{itemize}

\subsection{Описание игры}

Игрок управляет подростком, который случайно стал обладателем важной информации против преступного синдиката. Теперь за ним охотятся бандиты, и ему нужно выбраться из города, чтобы выжить и передать информацию. На каждом уровне игрок сталкивается с различными препятствиями, такими как патрули бандитов, ловушки и разрушаемые объекты.

\subsection{Предпосылки создания}

\begin{itemize}
    \item \textbf{Рыночные тенденции}: Популярность платформеров с элементами стелса и экшена растет, особенно среди молодежи.
    \item \textbf{Лицензирование}: Игра не использует лицензионные материалы.
\end{itemize}

\subsection{Платформа}

\begin{itemize}
    \item \textbf{Платформы}: PC.
    \item \textbf{Системные требования}:
\end{itemize}

\begin{center}
\begin{tabular}{|l|l|l|}
\hline
Требования              & Минимальные         & Рекомендуемые         \\ \hline
Операционная система    & Windows 10          & Windows 10/11         \\ \hline
Процессор               & Intel i5-4460       & Intel i7-7700K        \\ \hline
ОЗУ                     & 2 ГБ                & 4 ГБ                 \\ \hline
Видеокарта              & NVIDIA GTX 960      & NVIDIA GTX 1070       \\ \hline
Свободное место на HDD  & 10 ГБ               & 15 ГБ                 \\ \hline
\end{tabular}
\end{center}

\newpage

\section{Функциональная спецификация}

\subsection{Принципы игры}

\subsubsection{Суть игрового процесса}

Игрок управляет подростком, который должен пройти через серию уровней, избегая врагов и преодолевая препятствия. Основная цель — добраться до конца уровня, не будучи пойманным. Игрок может использовать укрытия, ловушки и отвлекающие маневры, чтобы избежать преследователей.

\subsubsection{Ход игры и сюжет}

Игра начинается с вступительного экрана, где игрок узнает о ситуации, в которой оказался главный герой. Каждый уровень представляет собой часть города, где игрок должен избегать патрулей бандитов и находить путь к спасению. По мере прохождения уровней игрок узнает больше о преступном синдикате и его планах.

\subsection{Физическая модель}

\begin{itemize}
    \item \textbf{Перемещения}: Игрок может бегать, прыгать, приседать и использовать укрытия.
    \item \textbf{Боевые действия}: Игрок не может напрямую сражаться с врагами, но может использовать окружение для создания ловушек и отвлекающих маневров.
    \item \textbf{Общие формулы}:
    \begin{itemize}
        \item Скорость движения: \( v = a \cdot t \)
        \item Высота прыжка: \( h = v_0 \cdot t - 0.5 \cdot g \cdot t^2 \)
    \end{itemize}
\end{itemize}

\subsection{Персонаж игрока}

Главный герой — подросток 16 лет, который случайно стал свидетелем преступления. Он не обладает боевыми навыками, но умеет быстро бегать и прятаться.

\subsection{Элементы игры}

\begin{itemize}
    \item \textbf{Укрытия}: Игрок может прятаться в укрытиях, чтобы избежать обнаружения.
    \item \textbf{Препятствия}: Разрушаемые объекты, платформы и другие препятствия.
\end{itemize}

\subsection{Искусственный интеллект}

\begin{itemize}
    \item \textbf{Поведение врагов}: Враги патрулируют уровни и стреляют в сторону игрока.
\end{itemize}

\subsection{Интерфейс пользователя}

\subsubsection{Блок-схема}

[Блок-схема интерфейса]

\subsubsection{Функциональное описание и управление игровым процессом}

\begin{itemize}
    \item \textbf{Главное меню}:
    \begin{itemize}
        \item Кнопка "Start": запускает новую игру.
        \item Кнопка "Continue": загружает сохраненный прогресс.
        \item Кнопка "Exit": закрывает игру.
    \end{itemize}
    \item \textbf{Экран игры}:
    \begin{itemize}
        \item Управление: стрелки влево/вправо, прыжок (пробел).
    \end{itemize}
\end{itemize}

\subsubsection{Объекты интерфейса пользователя}

\begin{itemize}
    \item \textbf{Главное меню}: Начальный экран с опциями "Играть", "Продолжить", "Выход".
\end{itemize}

\newpage

\section{4. Контакты}

\begin{itemize}
    \item \textbf{Контактное лицо}: Скачков Юрий
    \item \textbf{Телефон}: 8962*******
    \item \textbf{E-mail}: skachkov-yuriy@mail.ru
\end{itemize}

\end{document}